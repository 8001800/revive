\documentclass[12pt]{article}
\usepackage{titling}

\setlength{\droptitle}{-12em}
\title{Payment Subnetworks\vspace{-5em}}
\date{}
\author{}

\begin{document}
\maketitle
\begin{abstract}
Two-way payment channels have been designed and linked together to construct payment networks to allow payment routing between parties that are not directly connected. A problem that members of this architecture might face is the need to rebalance ledgers. Forming a subnetwork of payment channels which allows for off-chain simultaneous rebalancing of connected channels would alleviate the problem. Implementation of this concept on the Ethereum block chain rather than Bitcoin can greatly simplify the model.
\end{abstract}

The design for bitcoin proposes using revocable secrets to proceed with off-chain payments. A design for Ethereum can make use of signatures and round numbers to greatly simplify off-chain interactions and dispute settlement.

In the Bitcoin proposition, altering the membership of nodes to subnetworks\footnote{Named Channel Factories in the Bitcoin paper.} requires the collaboration of all members of all subnetworks to sign a new first layer allocation. In Ethereum, using smart contracts, this requirement can be easily annulled, and subnetwork membership changed would only require the approval of the members affected subnetwork.


\end{document}